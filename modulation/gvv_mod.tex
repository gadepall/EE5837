\documentclass[journal,12pt,twocolumn]{IEEEtran}
%
\usepackage{setspace}
\usepackage{gensymb}
\usepackage{xcolor}
\usepackage{caption}
%\usepackage{subcaption}
%\doublespacing
\singlespacing
\usepackage{multicol}

\usepackage{iithtlc}
%\usepackage{graphicx}
%\usepackage{amssymb}
%\usepackage{relsize}
\usepackage[cmex10]{amsmath}
\usepackage{mathtools}
%\usepackage{amsthm}
%\interdisplaylinepenalty=2500
%\savesymbol{iint}
%\usepackage{txfonts}
%\restoresymbol{TXF}{iint}
%\usepackage{wasysym}
\usepackage{amsthm}
\usepackage{mathrsfs}
\usepackage{txfonts}
\usepackage{stfloats}
\usepackage{cite}
\usepackage{cases}
\usepackage{subfig}
%\usepackage{xtab}
\usepackage{longtable}
\usepackage{multirow}
%\usepackage{algorithm}
%\usepackage{algpseudocode}
\usepackage{enumitem}
\usepackage{mathtools}
%\usepackage{stmaryrd}

\usepackage{listings}
    \usepackage[latin1]{inputenc}                                 %%
    \usepackage{color}                                            %%
    \usepackage{array}                                            %%
    \usepackage{longtable}                                        %%
    \usepackage{calc}                                             %%
    \usepackage{multirow}                                         %%
    \usepackage{hhline}                                           %%
    \usepackage{ifthen}                                           %%
  %optionally (for landscape tables embedded in another document): %%
    \usepackage{lscape}     

%\usepackage{wasysym}
%\newcounter{MYtempeqncnt}
\DeclareMathOperator*{\Res}{Res}
%\renewcommand{\baselinestretch}{2}
\renewcommand\thesection{\arabic{section}}
\renewcommand\thesubsection{\thesection.\arabic{subsection}}
\renewcommand\thesubsubsection{\thesubsection.\arabic{subsubsection}}

\renewcommand\thesectiondis{\arabic{section}}
\renewcommand\thesubsectiondis{\thesectiondis.\arabic{subsection}}
\renewcommand\thesubsubsectiondis{\thesubsectiondis.\arabic{subsubsection}}

% correct bad hyphenation here
\hyphenation{op-tical net-works semi-conduc-tor}

\def\inputGnumericTable{}  

\lstset{
language=python,
frame=single, 
breaklines=true
}

\begin{document}
%

\theoremstyle{definition}

\newtheorem{theorem}{Theorem}[section]
\newtheorem{problem}{Problem}
\newtheorem{proposition}{Proposition}[section]
\newtheorem{lemma}{Lemma}[section]
\newtheorem{corollary}[theorem]{Corollary}
\newtheorem{example}{Example}[section]
\newtheorem{definition}{Definition}[section]
%\newtheorem{algorithm}{Algorithm}[section]
%\newtheorem{cor}{Corollary}
\newcommand{\BEQA}{\begin{eqnarray}}
\newcommand{\EEQA}{\end{eqnarray}}
\newcommand{\define}{\stackrel{\triangle}{=}}

\bibliographystyle{IEEEtran}
%\bibliographystyle{ieeetr}



\providecommand{\pr}[1]{\ensuremath{\Pr\left(#1\right)}}
\providecommand{\qfunc}[1]{\ensuremath{Q\left(#1\right)}}
\providecommand{\sbrak}[1]{\ensuremath{{}\left[#1\right]}}
\providecommand{\lsbrak}[1]{\ensuremath{{}\left[#1\right.}}
\providecommand{\rsbrak}[1]{\ensuremath{{}\left.#1\right]}}
\providecommand{\brak}[1]{\ensuremath{\left(#1\right)}}
\providecommand{\lbrak}[1]{\ensuremath{\left(#1\right.}}
\providecommand{\rbrak}[1]{\ensuremath{\left.#1\right)}}
\providecommand{\cbrak}[1]{\ensuremath{\left\{#1\right\}}}
\providecommand{\lcbrak}[1]{\ensuremath{\left\{#1\right.}}
\providecommand{\rcbrak}[1]{\ensuremath{\left.#1\right\}}}
\theoremstyle{remark}
\newtheorem{rem}{Remark}
\newcommand{\sgn}{\mathop{\mathrm{sgn}}}
\providecommand{\abs}[1]{\left\vert#1\right\vert}
\providecommand{\res}[1]{\Res\displaylimits_{#1}} 
\providecommand{\norm}[1]{\lVert#1\rVert}
\providecommand{\mtx}[1]{\mathbf{#1}}
\providecommand{\mean}[1]{E\left[ #1 \right]}
\providecommand{\fourier}{\overset{\mathcal{F}}{ \rightleftharpoons}}
%\providecommand{\hilbert}{\overset{\mathcal{H}}{ \rightleftharpoons}}
\providecommand{\system}{\overset{\mathcal{H}}{ \longleftrightarrow}}
\providecommand{\gauss}[2]{\mathcal{N}\ensuremath{\left(#1,#2\right)}}
	%\newcommand{\solution}[2]{\textbf{Solution:}{#1}}
\newcommand{\solution}{\noindent \textbf{Solution: }}
\providecommand{\dec}[2]{\ensuremath{\overset{#1}{\underset{#2}{\gtrless}}}}
%\numberwithin{equation}{section}
%\numberwithin{problem}{section}

\def\putbox#1#2#3{\makebox[0in][l]{\makebox[#1][l]{}\raisebox{\baselineskip}[0in][0in]{\raisebox{#2}[0in][0in]{#3}}}}
     \def\rightbox#1{\makebox[0in][r]{#1}}
     \def\centbox#1{\makebox[0in]{#1}}
     \def\topbox#1{\raisebox{-\baselineskip}[0in][0in]{#1}}
     \def\midbox#1{\raisebox{-0.5\baselineskip}[0in][0in]{#1}}


% paper title
% can use linebreaks \\ within to get better formatting as desired
\title{
\logo{
Applied Probability: Digital Communication 
}
}
%
%
% author names and IEEE memberships
% note positions of commas and nonbreaking spaces ( ~ ) LaTeX will not break
% a structure at a ~ so this keeps an author's name from being broken across
% two lines.
% use \thanks{} to gain access to the first footnote area
% a separate \thanks must be used for each paragraph as LaTeX2e's \thanks
% was not built to handle multiple paragraphs
%

%\author{Y Aditya, A Rathnakar and G V V Sharma$^{*}$% <-this % stops a space
\author{G V V Sharma$^{*}$% <-this % stops a space
\thanks{*The author is with the Department
of Electrical Engineering, Indian Institute of Technology, Hyderabad
502205 India e-mail:  gadepall@iith.ac.in.}% <-this % stops a space
%\thanks{J. Doe and J. Doe are with Anonymous University.}% <-this % stops a space
%\thanks{Manuscript received April 19, 2005; revised January 11, 2007.}}
}



% make the title area
\maketitle

\tableofcontents

\bigskip

\renewcommand{\thefigure}{\theenumi}
\renewcommand{\thetable}{\theenumi}

\begin{abstract}
%\boldmath
The manual frames the problems of receiver design and performance analysis in digital communication as applications of probability theory.

\end{abstract}
% IEEEtran.cls defaults to using nonbold math in the Abstract.
% This preserves the distinction between vectors and scalars. However,
% if the journal you are submitting to favors bold math in the abstract,
% then you can use LaTeX's standard command \boldmath at the very start
% of the abstract to achieve this. Many IEEE journals frown on math
% in the abstract anyway.

% Note that keywords are not normally used for peerreview papers.
%\begin{IEEEkeywords}
%Cooperative diversity, decode and forward, piecewise linear
%\end{IEEEkeywords}



% For peer review papers, you can put extra information on the cover
% page as needed:
% \ifCLASSOPTIONpeerreview
% \begin{center} \bfseries EDICS Category: 3-BBND \end{center}
% \fi
%
% For peerreview papers, this IEEEtran command inserts a page break and
% creates the second title. It will be ignored for other modes.
\IEEEpeerreviewmaketitle


\section{Multivariate Gaussian}
\begin{enumerate}[label=\thesection.\arabic*
,ref=\thesection.\theenumi]

\item The multivariate Gaussian distribution is defined as
%
\begin{multline}
p_{\mathbf{x}}(x_1,\dots,x_k)
\\
=\frac{1}{\sqrt{\brak{2\pi}^k\abs{\Sigma}}}\exp\cbrak{-\frac{1}{2}\brak{\mathbf{x}-\mathbf{\mu}}^T\Sigma^{-1}\brak{\mathbf{x}-\mathbf{\mu}}}
\end{multline}
%
where $\mathbf{\mu}$ is the mean vector, $\Sigma = E\sbrak{\brak{\mathbf{x}-\mathbf{\mu}}\brak{\mathbf{x}-\mathbf{\mu}}^T}$ is the covariance matrix and $\abs{\Sigma}$ is the determinant of $\Sigma$.
\item Show that
\begin{multline}
p(x,y)= \frac{1}{2\pi \sigma_x\sigma_y\sqrt{1-\rho^2}}\exp\lsbrak{-\frac{1}{2\brak{1-\rho^2}}}
\\
\times \rsbrak{\cbrak{\frac{\brak{x-\mu_x}^2}{\sigma_x^2}+\frac{\brak{y-\mu_y}^2}{\sigma_y^2}-\frac{2\rho\brak{x-\mu_x}\brak{y-\mu_y}}{\sigma_x\sigma_y}}}
\end{multline}
%
where
%
\begin{align}
\mathbf{\mu}=
\begin{pmatrix*}
\mu_x \\
\mu_y
\end{pmatrix*},
\Sigma = 
\begin{pmatrix*}%[r]
\sigma_x^2 & \rho\sigma_x\sigma_y \\
\rho\sigma_x\sigma_y & \sigma_y^2
\end{pmatrix*}
\end{align}
%




%\subsection{Bivariate Gaussian}

%
 %Show that 
%
\item
If 
\begin{align}
\mathbf{y}|0 = 
\begin{pmatrix*}
\sqrt{A} + n_{1}\\
n_{2}
\end{pmatrix*},
\end{align}
and 
\begin{align}
\mathbf{y}|1 = 
\begin{pmatrix*}
n_{1}\\
\sqrt{A} + n_{2}
\end{pmatrix*},
\end{align}
use the MAP criterion to reach a decision.

\item
Derive and plot the probability of error.  Verify through simulation.

%
\end{enumerate}
\section{Coherent BFSK}
\begin{enumerate}[label=\thesection.\arabic*
,ref=\thesection.\theenumi]

\item
Let
\begin{equation}
\mathbf{r} = \mathbf{s}+ \mathbf{n}
\end{equation}
where $\mathbf{s} \in \cbrak{s_0,s_1,s_2, s_3}$ and
\begin{align}
\mathbf{s}_0 &= 
\begin{pmatrix*}
A\\
0
\end{pmatrix*},
\mathbf{s}_1 = 
\begin{pmatrix*}
0\\
A
\end{pmatrix*},
\mathbf{s}_2 = 
\begin{pmatrix*}
-A\\
0
\end{pmatrix*},
\mathbf{s}_3 = 
\begin{pmatrix*}
0\\
-A
\end{pmatrix*},
\\
E\sbrak{\mathbf{n}} &= \mathbf{0}, E\sbrak{\mathbf{n}\mathbf{n}^T} = \sigma^2 \mathbf{I}
\end{align}
%
\begin{enumerate}
\item Show that the MAP decision for detecting $\mathbf{s}_0$ results in
\begin{equation}
\abs{r}_2 < r_1
\end{equation}
\item Express $\pr{\hat{\mathbf{s}} = \mathbf{s}_0|\mathbf{s} = \mathbf{s}_0}$ in terms of $r_1, r_2$.
Let $X=n_2-n_1, Y = -n_2-n_1$, where $\mathbf{n}=\brak{n_1,n_2}$.
Their correlation coefficient is defined as
%
\begin{align}
\rho = \frac{E\sbrak{\brak{X-\mu_x}\brak{Y-\mu_y}}}{\sigma_x\sigma_y}
\end{align}
%
$X$ and $Y$ are said to be uncorrelated if $\rho = 0$
\item Show that if $X$ and $Y$ are uncorrelated 
Verify this numerically.
\item Show that $X$ and $Y$ are independent, i.e. $p_{XY}(x,y) = p_{X}(x)p_{Y}(y)$.
\item Show that $X,Y \sim \mathcal{N}\brak{0,2\sigma^2}$.
\item Show that $\pr{\hat{\mathbf{s}} = \mathbf{s}_0|\mathbf{s} = \mathbf{s}_0} =\pr{ X < A,  Y < A}$.
\item Find $\pr{ X < A,  Y < A}$.
\item Verify the above through simulation.
\end{enumerate}

\end{enumerate}
\section{Noncoherent BFSK}
\begin{enumerate}[label=\thesection.\arabic*
,ref=\thesection.\theenumi]

\item
Show that
%
\begin{align}
I_{0}(x) &= \frac{1}{2\pi}\int_{0}^{2\pi}e^{x\cos\theta}\,d\theta \\
I_{0}(x) &= \frac{1}{2\pi}\int_{0}^{2\pi}e^{x\cos\brak{\theta-\phi}}\,d\theta \\
\frac{1}{2\pi}\int_{0}^{2\pi}e^{m_1\cos\theta + m_2\sin\theta}\,d\theta &= I_0\brak{\sqrt{m_1^2+m_2^2}} 
\end{align}
%
where the modified Bessel function of order $n$ (integer) is defined as 
%
\begin{align}
I_{n}(x) = \frac{1}{\pi}\int_{0}^{\pi}e^{x\cos\theta}\cos n\theta\,d\theta
\end{align}

\item
Let
%
\begin{align}
\mathbf{r}|0= \sqrt{E_b}
\begin{pmatrix}
\cos \phi_0\\
\sin \phi_0 \\
0\\
0
\end{pmatrix}
+\mathbf{n}_0,
\mathbf{r}|1= \sqrt{E_b}
\begin{pmatrix}
0\\
0 \\
\cos \phi_1\\
\sin \phi_1 
\end{pmatrix}
+\mathbf{n}_1
\end{align}
%
where $\mathbf{n}_0,\mathbf{n}_1\sim \mathcal{N}\brak{\mathbf{0}, \frac{N_0}{2}\mathbf{I}}$.
%
\begin{enumerate}
\item Taking $\mathbf{r} = \brak{r_1,r_2,r_3,r_4}^{T},$, find the pdf $p\brak{\mathbf{r}|0,\phi_0}$ in
terms of $r_1,r_2,r_3,r_4,\phi,E_b$ and $N_0$. Assume that all noise variables are independent.
%
\item 
If $\phi_0$ is uniformly distributed between 0 and $2\pi$, find $p\brak{\mathbf{r}|0}$.  Note that this expression will no longer contain $\phi_0$.
%
\item
Show that the ML detection criterion for this scheme is
%
\begin{align}
I_0\brak{k\sqrt{r_1^2+r_2^2}}\dec{0}{1}I_0\brak{k\sqrt{r_3^2+r_4^2}}
\end{align}
%
where $k$ is a constant.
%
\item 
The above criterion reduces to something simpler.  Can you guess what it is?  Justify your answer.
%
\item 
Show that 
%
\begin{align}
P_{e|0}=\pr{r_1^2+r_2^2 < r_3^2+r_4^2 | 0}
\end{align}
%
\item 
Show that the pdf of $Y=r_3^2+r_4^2$ id
%
\begin{align}
p_{Y}(y) = \frac{1}{N_0}e^{-\frac{y}{N_0}}, y > 0
\end{align}
%
\item 
Find 
%
\begin{align}
g\brak{r_1,r_2} = \pr{r_1^2+r_2^2<Y|0,r_1,r_2}.
\end{align}
\item 
Show that $E\sbrak{e^{-\frac{X^2}{2\sigma^2}}}=\frac{1}{\sqrt{2}}e^{-\frac{\mu^2}{4\sigma^2}}$ for $X \sim 
\mathcal{N}\brak{\mu,\sigma^2}$.
%
\item 
Now show that
%
\begin{align}
E\sbrak{g\brak{r_1,r_2}}=\frac{1}{2}e^{-\frac{E_b}{2N_0}}.
\end{align}
%
\end{enumerate}

\item
 Let $U,V\sim\mathcal{N}\brak{0,\frac{k}{2}}$ be i.i.d.  Assuming that
%
\begin{align}
U = \sqrt{R} \cos \Theta \\
V = \sqrt{R} \sin \Theta
\end{align}
\begin{enumerate}
\item 
Compute the jacobian for $U,V$ with respect to $X$ and $\Theta$ defined by
%
\begin{align}
J = \det\brak{
\begin{matrix}
\frac{\partial U}{\partial R} & \frac{\partial U}{\partial \Theta} \\
\frac{\partial V}{\partial R} & \frac{\partial V}{\partial \Theta}
\end{matrix}
}
\end{align}
\item 
The joint pdf for $R,\Theta$ is given by,
%
\begin{align}
p_{R,\Theta}\brak{r,\theta} = p_{U,V}\brak{u,v}J\vert_{u = \sqrt{r}\cos\theta,v = \sqrt{r}\sin\theta}
\end{align}
%
Show that
%
\begin{align}
p_{R}(r) = 
\begin{cases}
\frac{1}{k}e^{-\frac{r}{k}} & r > 0, \\
0 & r < 0,
\end{cases}
\end{align}
%
assuming that $\Theta$ is uniformly distributed between 0 to $2\pi$.
\item
Show that the pdf of $Y = R_1-R_2$, where $R_1$ and $R_2$ are i.i.d. and have the same distribution as $R$ is
%
\begin{align}
p_{Y}(y) = \frac{1}{2k}e^{-\frac{\abs{y}}{k}}
\end{align}
%
\item 
 Find the pdf of 
%
\begin{align}
Z = p + \sqrt{p}\sbrak{U \cos \phi + V \sin \phi}
\end{align}
%
where $\phi$ is a constant.
\item 
Find $\pr{Y > Z}$.
\item 
If $U\sim\mathcal{N}\brak{m_1,\frac{k}{2}},V\sim\mathcal{N}\brak{m_2,\frac{k}{2}}$, where $m_1,m_2, k$ are constants, show that the pdf of 
%
\begin{align}
R = \sqrt{U^2+V^2}
\end{align}
%
is
%
\begin{align}
p_{R}\brak{r} = \frac{e^{-\frac{r +m}{k}}}{ k}I_{0}\brak{\frac{2\sqrt{mr}}{k}},\quad m = \sqrt{m_1^2+m_2^2}
\end{align}
%
\item
Show that
\begin{align}
I_0(x) = \sum_{n=0}^{\infty}\frac{x^{2n}}{4^n\brak{n!}^2}
\end{align}
\item 
If
%
\begin{align}
p_{Z}(z) &= 
\begin{cases}
\frac{1}{k} e^{-\frac{z}{k}} & z \geq 0 \\
0 & z < 0
\end{cases}
\end{align}
%
find $\pr{R < Z}$.
\end{enumerate}

\end{enumerate}
\section{$M$-PSK}
\begin{enumerate}[label=\thesection.\arabic*
,ref=\thesection.\theenumi]

\item
Consider a system where 
$\mathbf{s}_i=
\begin{pmatrix}
\cos\brak{\frac{2\pi i}{M}}\\
\cos\brak{\frac{2\pi i}{M}}
\end{pmatrix}, i = 0, 1 , \dots M-1
$
.
Let
%
\begin{align}
\mathbf{r}|s_0 = 
\begin{pmatrix}
r_1\\
r_2
\end{pmatrix}
=
\begin{pmatrix}
\sqrt{E_s}+n_1\\
n_2
\end{pmatrix}
\end{align}
where $n_1,n_2 \sim \mathcal{N}\brak{0,\frac{N_0}{2}}$.

\begin{enumerate}
\item Substituting 
\begin{align}
r_1=R\cos \theta \\
r_2=R\sin \theta
\end{align}
show that the joint pdf of $R,\theta$ is
%
\begin{equation}
p\brak{R,\theta}=\frac{R}{\pi N_0}\exp\brak{-\frac{R^2-2R\sqrt{E_s}\cos \theta + E_s}{N_0}}
\end{equation}
%
\item Show that 
%
\begin{align}
\lim_{\alpha \rightarrow \infty}\int_{0}^{\infty}\brak{V-\alpha }e^{-\brak{V-\alpha}^2 }\,dV
&= 0
\\
\lim_{\alpha \rightarrow \infty}\int_{0}^{\infty} e^{-\brak{V-\alpha}^2 }\,dV
&=  \sqrt{\pi}
\end{align}
%
\item 
Using the above, evaluate
%
\begin{align}
\int_{0}^{\infty}V\exp\cbrak{-\brak{V^2 - 2V \sqrt{\gamma}\cos \theta +\gamma}}\,dV
\end{align}
%
for large values of $\gamma$.
\item
Find a compact expression for 
%
\begin{align}
I = 1 - \sqrt{\frac{\gamma}{\pi}}\int_{-\frac{\pi}{M}}^{\frac{\pi}{M}}e^{- \gamma\sin^2\theta }\cos \theta\, d\theta
\end{align}
\item Find $P_{e|\mathbf{s}_0}$.
\end{enumerate}

%
\end{enumerate}
\section{Craig's Formula and MGF}
\begin{enumerate}[label=\thesection.\arabic*
,ref=\thesection.\theenumi]

\item
The Moment Generating Function (MGF) of $X$ is defined as
%
\begin{align}
M_{X}(s) = E\sbrak{e^{s X}}
\end{align}
%
where $X$ is a random variable and $E\sbrak{\cdot}$ is the expectation.  
%
%
\begin{enumerate}
\item Let $Y \sim \gauss{0}{1}$.  Define
%
\begin{align}
Q(x) = \pr{Y > x}, x > 0
\end{align}
%
Show that
\begin{equation}
Q(x) = \frac{1}{\pi}\int^{\frac{\pi}{2}}_{0}e^{-\frac{x^2}{2\sin^2 \theta}}\,d\theta
\end{equation}
\item 
Let $h\sim\mathcal{CN}\brak{0,\frac{\Omega}{2}},n\sim\mathcal{CN}\brak{0,\frac{N_0}{2}}$.  Find the distribution of $\abs{h}^2$.

\item Let
%
\begin{align}
P_e = \pr{\Re \cbrak{h^*y} < 0}, \text{ where } y = \brak{\sqrt{E_s}h + n},
\end{align}
%
Show that
%
\begin{align}
P_e = \int_{0}^{\infty}\qfunc{\sqrt{2x}}p_{A}(x) \,dx
\end{align}
where $A = \frac{E_s\abs{h}^2}{N_0}$.
\item Show that
%
\begin{align}
P_e 
%&= E\sbrak{\qfunc{\sqrt{2A}}} \\
%&=  \frac{1}{\pi}\int_{0}^{\frac{\pi}{2}}E\sbrak{e^{-\frac{A}{\sin^2\theta}}}\,d\theta \\
&=  \frac{1}{\pi}\int_{0}^{\frac{\pi}{2}}M_{A}\brak{-\frac{1}{\sin^2\theta}}\,d\theta
\label{ch4_pe_mgf}
\end{align}
%
\item compute $M_A(s)$.
%
\item 
Find $P_e$.
\item 
If $\gamma = \frac{\Omega E_s}{N_0}$, show that $P_e < \frac{1}{2\gamma}$. 
\end{enumerate}





%\newpage
%\section{Two Variable}
%
%

%
%\newpage
%\section{Application}
%\input{./chapter2/ch2}
%\newpage
%\section{Binary Modulation}
%\input{chapter2} 
%
%\newpage
%\section{$M$-ary Modulation}
%\input{chapter3} 

%\newpage
%\section{BER in Rayleigh Flat Slowly Fading Channels}
%\input{chapter4} 
\end{enumerate}
\end{document}


