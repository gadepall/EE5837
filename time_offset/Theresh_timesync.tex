\documentclass[journal,12pt,twocolumn]{IEEEtran}
%
\makeatletter
\makeatother
\usepackage{setspace}
\usepackage{gensymb}
\usepackage{xcolor}
\usepackage{caption}
%\usepackage{stackengine}
%\usepackage{subcaption}
%\doublespacing
\singlespacing



\usepackage{graphicx}
\graphicspath{ {./images}  }
%\usepackage{amssymb}
%\usepackage{relsize}
\usepackage[cmex10]{amsmath}
\usepackage{mathtools}
%\usepackage{amsthm}
%\interdisplaylinepenalty=2500
%\savesymbol{iint}
%\usepackage{txfonts}
%\restoresymbol{TXF}{iint}
\usepackage{wasysym}
\usepackage{amsthm}
\usepackage{mathrsfs}
\usepackage{txfonts}
\usepackage{stfloats}
\usepackage{cite}
\usepackage{cases}
\usepackage{mathtools}
\usepackage{subfig}
\usepackage{enumerate}	
\usepackage{enumitem}
\usepackage{amsmath}
%\usepackage{xtab}
\usepackage{longtable}
\usepackage{multirow}
%\usepackage{algorithm}
%\usepackage{algpseudocode}
\usepackage{enumitem}
\usepackage{mathtools}
%\usepackage{iithtlc}
%\usepackage[framemethod=tikz]{mdframed}
\usepackage{listings}
\usepackage{listings}
    \usepackage[latin1]{inputenc}                                 %%
    \usepackage{color}                                            %%
    \usepackage{array}                                            %%
    \usepackage{longtable}                                        %%
    \usepackage{calc}                                             %%
    \usepackage{multirow}                                         %%
    \usepackage{hhline}                                           %%
    \usepackage{ifthen}                                           %%
  %optionally (for landscape tables embedded in another document): %%
    \usepackage{lscape}     



%\usepackage{stmaryrd}


%\usepackage{wasysym}
%\newcounter{MYtempeqncnt}
\DeclareMathOperator*{\Res}{Res}
%\renewcommand{\baselinestretch}{4}
%\setcounter{secnumdepth}{4}
\renewcommand\thesection{\arabic{section}}
\renewcommand\thesubsection{\thesection.\arabic{subsection}}
\renewcommand\thesubsubsection{\thesubsection.\arabic{subsubsection}}
%\renewcommand\thesubsubsubsection{\thesubsubsection.\arabic{subsubsubsection}}

%\renewcommand\thesectiondis{\arabic{section}}
%\renewcommand\thesubsectiondis{\thesectiondis.\arabic{subsection}}
%\renewcommand\thesubsubsectiondis{\thesubsectiondis.\arabic{subsubsection}}
%\renewcommand\thesubsubsubsectiondis{\thesubsubsectiondis.\arabic{subsubsubsection}}
% correct bad hyphenation here
\hyphenation{op-tical net-works semi-conduc-tor}

%\lstset{
%language=C,
%frame=single, 
%breaklines=true
%}

%\lstset{
	%%basicstyle=\small\ttfamily\bfseries,
	%%numberstyle=\small\ttfamily,
	%language=Octave,
	%backgroundcolor=\color{white},
	%%frame=single,
	%%keywordstyle=\bfseries,
	%%breaklines=true,
	%%showstringspaces=false,
	%%xleftmargin=-10mm,
	%%aboveskip=-1mm,
	%%belowskip=0mm
%}

%\surroundwithmdframed[width=\columnwidth]{lstlisting}
\def\inputGnumericTable{}                                 %%
\lstset{
language=C,
frame=single, 
breaklines=true
}
 

\begin{document}
%

\theoremstyle{definition}
\newtheorem{theorem}{Theorem}[section]
\newtheorem{problem}{Problem}
\newtheorem{proposition}{Proposition}[section]
\newtheorem{lemma}{Lemma}[section]
\newtheorem{corollary}[theorem]{Corollary}
\newtheorem{example}{Example}[section]
\newtheorem{definition}{Definition}[section]
%\newtheorem{algorithm}{Algorithm}[section]
%\newtheorem{cor}{Corollary}
\newcommand{\BEQA}{\begin{eqnarray}}
\newcommand{\EEQA}{\end{eqnarray}}
\newcommand{\define}{\stackrel{\triangle}{=}}

\bibliographystyle{IEEEtran}
%\bibliographystyle{ieeetr}

\providecommand{\nCr}[2]{\,^{#1}C_{#2}} % nCr
\providecommand{\nPr}[2]{\,^{#1}P_{#2}} % nPr
\providecommand{\mbf}{\mathbf}
\providecommand{\pr}[1]{\ensuremath{\Pr\left(#1\right)}}
\providecommand{\qfunc}[1]{\ensuremath{Q\left(#1\right)}}
\providecommand{\sbrak}[1]{\ensuremath{{}\left[#1\right]}}
\providecommand{\lsbrak}[1]{\ensuremath{{}\left[#1\right.}}
\providecommand{\rsbrak}[1]{\ensuremath{{}\left.#1\right]}}
\providecommand{\brak}[1]{\ensuremath{\left(#1\right)}}
\providecommand{\lbrak}[1]{\ensuremath{\left(#1\right.}}
\providecommand{\rbrak}[1]{\ensuremath{\left.#1\right)}}
\providecommand{\cbrak}[1]{\ensuremath{\left\{#1\right\}}}
\providecommand{\lcbrak}[1]{\ensuremath{\left\{#1\right.}}
\providecommand{\rcbrak}[1]{\ensuremath{\left.#1\right\}}}
\theoremstyle{remark}
\newtheorem{rem}{Remark}
\newcommand{\sgn}{\mathop{\mathrm{sgn}}}
\providecommand{\abs}[1]{\left\vert#1\right\vert}
\providecommand{\res}[1]{\Res\displaylimits_{#1}} 
\providecommand{\norm}[1]{\lVert#1\rVert}
\providecommand{\mtx}[1]{\mathbf{#1}}
\providecommand{\mean}[1]{E\left[ #1 \right]}
\providecommand{\fourier}{\overset{\mathcal{F}}{ \rightleftharpoons}}
%\providecommand{\hilbert}{\overset{\mathcal{H}}{ \rightleftharpoons}}
\providecommand{\system}{\overset{\mathcal{H}}{ \longleftrightarrow}}
	%\newcommand{\solution}[2]{\textbf{Solution:}{#1}}
\newcommand{\solution}{\noindent \textbf{Solution: }}
\providecommand{\dec}[2]{\ensuremath{\overset{#1}{\underset{#2}{\gtrless}}}}
\DeclarePairedDelimiter{\ceil}{\lceil}{\rceil}
%\numberwithin{equation}{subsection}
\numberwithin{equation}{problem}
%\numberwithin{problem}{subsection}
%\numberwithin{definition}{subsection}
\makeatletter
\@addtoreset{figure}{problem}
\makeatother

\let\StandardTheFigure\thefigure
%\renewcommand{\thefigure}{\theproblem.\arabic{figure}}
\renewcommand{\thefigure}{\theproblem}


%\numberwithin{figure}{subsection}

%\numberwithin{equation}{subsection}
%\numberwithin{equation}{section}
\numberwithin{equation}{problem}
%\numberwithin{problem}{subsection}
\numberwithin{problem}{section}
%%\numberwithin{definition}{subsection}
%\makeatletter
%\@addtoreset{figure}{problem}
%\makeatother
\makeatletter
\@addtoreset{table}{problem}
\makeatother

\let\StandardTheFigure\thefigure
\let\StandardTheTable\thetable
%%\renewcommand{\thefigure}{\theproblem.\arabic{figure}}
%\renewcommand{\thefigure}{\theproblem}

%%\numberwithin{figure}{section}

%%\numberwithin{figure}{subsection}



\def\putbox#1#2#3{\makebox[0in][l]{\makebox[#1][l]{}\raisebox{\baselineskip}[0in][0in]{\raisebox{#2}[0in][0in]{#3}}}}
     \def\rightbox#1{\makebox[0in][r]{#1}}
     \def\centbox#1{\makebox[0in]{#1}}
     \def\topbox#1{\raisebox{-\baselineskip}[0in][0in]{#1}}
     \def\midbox#1{\raisebox{-0.5\baselineskip}[0in][0in]{#1}}



\title{ 
%	\logo{
Timing Offset Synchronization using Gardner Timing Error Detector (TED) Algorithm
%	}
}



\author{Theresh Babu,Sandeep Kumar and G V V Sharma$^{*}$% <-this % stops a space
\thanks{*The authors are with the Department
of Electrical Engineering, Indian Institute of Technology, Hyderabad
502285 India e-mail:  gadepall@iith.ac.in.}
}


% make the title area
\maketitle

\tableofcontents

%\bigskip
%
%\begin{abstract}
%%\boldmath
%A brief description about the modulation/demodulation blocks and Coding/Decoding blocks for DVBS2.
%% and the Kaiser window is used for the FIR filter.
%\end{abstract}

%\IEEEpeerreviewmaketitle
%
\section{Gardner TED}
%\subsection{Transmitter}
%\begin{equation}
%m \rightarrow\boxed	{BPSK}\rightarrow\boxed	{\uparrow T_{sym}}\rightarrow C	
%\end{equation}
%
%  Let bit stream was $m$, $xc$ will be mapped sequence and $C$ will be upsampled by $T_{sym}$
% Where $T_{sym}$ is the samples per symbol. 
%\begin{equation}
%X=P \circledast C
%\end{equation} where $P$ is the shape of the pulse. And defined as,
%\begin{equation}
%    P =
%    \begin{cases}
%      1 & 0\leq t\leq 99 \\
%      0        & otherwise
%    \end{cases}
%  \end{equation}
%\subsection{Receiver}
Let the $m$th sample in the $r$th received symbol time slot be
\begin{equation}
Y_r(m)= X_r + V_r(m), \quad r = 1,\dots,N, m = 1 ,\dots,M.
\end{equation} 
where $X_r$ is the $m$th transmitted symbol in the $r$th slot and $V_r(m) \sim \mathcal{N}\brak{0,\sigma^2} $. 
The decision variable for the $r$th symbol # -*- coding: utf-8 -*-
"""
Created on Fri Mar  8 15:22:28 2019

@author: theresh
"""
import numpy as np
from matplotlib import pyplot as plt
from scipy import special

N=100000
data=np.random.randint(0,2,N)
c=2*data-1
Tsym=100
#sym_samp=np.arange(0,Tsym)
Eb_N0_dB=np.linspace(0,8,9)
BER_sim=np.zeros(len(Eb_N0_dB))
pulse=np.ones(Tsym)
inter_data=np.zeros(N*Tsym)
for i in range(N):
    inter_data[i*Tsym:(i+1)*Tsym]=c[i]
#plt.plot(inter_data)
#plt.show()

#toff=np.zeros(len(Eb_N0_dB))
#poff=np.zeros(len(Eb_N0_dB))
for i in range(len(Eb_N0_dB)):
    N0 = 1/(np.exp(Eb_N0_dB[i]*np.log(10)/10.0))
    noise=np.random.normal(0,np.sqrt(N0/2.0),len(inter_data))
    rx=inter_data+noise
    tau=0
    delta=int(Tsym/2)
    center=60   # A r b i t a r y  r e c e i v e d  c e n t e r index
    #toff[i]=center%delta
    a=np.zeros(N)    
    avgsamples=6
    stepsize=1
    rit=-1
    GA=np.zeros(avgsamples)
    for k in range((delta),len(rx)-(delta),Tsym):
        #print k,
        rit=rit+1
        #b.append(rit)
        midsample=rx[center-1]
        latesample=rx[center+delta-1]
        earlysample=rx[center-delta-1]
        a[rit]=earlysample
        sub=latesample-earlysample
        GA[np.mod(rit,avgsamples)]=sub*midsample
        if (np.mean(GA)>0):
            tau=-stepsize
        else:
            tau=stepsize
        center=center+Tsym+tau
        if center>=len(rx)-delta:
            break;
    #print center
    #dum=np.fmod(center,delta)
    #poff[i]=dum
    xc=1*(a>0)
    error=(xc!=data).sum()
    BER_sim[i]=(1.0*error)/N 
    
theoryBer = 0.5*special.erfc(np.sqrt(10.0**(Eb_N0_dB/10.0)))    
plt.semilogy(Eb_N0_dB,theoryBer)    
plt.semilogy(Eb_N0_dB,BER_sim) 
plt.title("Timing Sync using Gardner TED")
plt.legend(['Theory','Simulated'],loc='best')
plt.xlabel('Eb/No (dB)')
plt.ylabel('$P_e$')
plt.grid(True)
plt.show()
is
\begin{align}
U_r&=Y_{r-1}\brak{\frac{M}{2}}\sbrak{Y_{r}\brak{M}-Y_{r-1}\brak{M}}
\end{align}

%\section{Derivation}
%\begin{align}
%U_t(r)&=L(r-1)-E(r)\\
%&=Y^2(\tau+(r-1)T_{sym})-Y^2(\tau+rT_{sym})\\
%&+2Y(\tau+(r-0.5)T_{sym})\cbrak{Y(\tau+rT_{sym})-Y(\tau+(r-1)T_{sym})}
%\end{align}
%Average over many samples,first two terms are equal.
%\begin{align}
%U_t(r)&=Y(\tau+(r-0.5)T_{sym})\cbrak{Y(\tau+rT_{sym})-Y(\tau+(r-1)T_{sym}}\\
%&=Y(r-0.5)\cbrak{Y(r)-Y(r-1)}
%\end{align}
%
%\begin{enumerate}
%\item Symbols are transmitted synchronously, spaced by the time
%interval T . Each sequence will have two samples per symbol
%interval and the samples will be time-coincident between the
%sequences. 
%\item One sample occurs at the data strobe time and the
%other sample occurs midway between data strobe times.
%The index r is used to designate symbol number. It is
%convenient to denote the strobe values of the rth symbol.
%pair of samples lying midway between the (r - 1)th and the
%rth strobes as yl(r - 1/2) and yQ(r - V2).
%A timing error detector operates upon samples and generates one error sample $u_t(r)$ for each symbol. 
%\end{enumerate}
\end{document}

